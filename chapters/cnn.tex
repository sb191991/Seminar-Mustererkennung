\chapter{Convolutional Neural Networks}
In den letzten Jahren haben Convolutional Neural Networks (CNN) in Zusammenhang mit Mustererkennung wie zum Beispiel Bildklassifizierung oder Spracherkennung bahnbrechende Ergebnisse erzielt \cite{bilderkennung}. Bei Bildklassifizierungen liefern CNNs die besten Ergebnisse \cite{imagenet}. Ein großer Vorteil bei CNNs ist, dass eine Reduzierung der Parameter stattfindet und dadurch größere Modelle mit komplexen Aufgaben im Anschluss besser klassifiziert werden können \cite{bilderkennung}. CNN ist ein hierarchisches neuronales Netz, welches aus unterschiedlichen Schichten (layers) besteht \cite{badshah2019deep}. Diese Schichten kann man in drei Hauptkomponenten aufteilen (siehe Abbildung \ref{architektur}.
\begin{itemize}
	\item  convolutional layers \newline diese Schicht ist für das Filtern des Inputs zuständig \cite{badshah2019deep}
	\item  pooling layers 
	\item  fully connected layers
	
\end{itemize}


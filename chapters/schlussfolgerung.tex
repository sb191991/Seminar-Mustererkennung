\chapter{Schlussfolgerung}

SER bietet ein breites Spektrum an Einsatzmöglichkeiten an und die Anwendungen von diesen Spracherkennungssystemen nehmen rasant zu. In der Automobilbranche kann so ein System bei der Erkennung des mentalen Zustands des Fahrers eine wichtige Rolle spielen \cite{badshah2019deep}. Des weiteren kann SER eine wichtige Komponente bei der Entwicklung intelligenter Dienste in der Gesundheitsversorgung, Audio-Forensik und Mensch-Maschine Interaktion sein \cite{badshah2019deep}.
\newline
Eigene Meinung:
 Das Problem, wie bei den meisten Machine Learning Algorithmen, ist es, gelabelte Daten zu finden, um ein Modell richtig trainieren zu können. Es ist unklar, ob die ausgewählten Emotionsklassen ausreichend und effektiv sind, um die Emotionen der Menschen wirklich charakterisieren zu können \cite{cnn}. Bei einem realen Anwendungsszenario mit einem SER-System müssen die Klassen durch weitere Emotionen ergänzt werden, damit bessere Ergebnisse erzielt werden können. Dies bedeutet wiederum, dass mehr Trainingsdatensätze gebraucht werden. Bei Menschen mit Emotionsstörungen ist es nicht einfach, solche SER-Systeme zu verwenden. Es bietet sich an SER-Systeme in weitere Systeme wie Gesichts- und Mimikerkennung zu adaptieren, um bessere Aussagen über Emotionen und Gefühle der jeweiligen Person treffen zu können.
\chapter*{Zusammenfassung}
Emotionen der Menschen spielen bei der Kommunikation und das Zusammenleben eine wichtige Rolle.
Intelligent personal assistants sind seit längerem ein Teil in unserem Leben und begleiten die Menschen durch den Alltag. Sprach-Interfaces kommen immer mehr zum Einsatz, sei es mit dem Handy oder bei Navigationssystemen im Auto.  Speech emotion recognition (SER) ist ein Teilgebiet der künstlichen Intelligenz, welche durch Mustererkennung die Emotionen der Menschen analysiert. Dabei werden Audiosignale in Form von Spektrogrammen untersucht, um anschließlich Aussagen über die Gefühle des Sprechers zu machen. In diesem Seminarbeit wird beschrieben, wie mit Hilfe von Convolutional Neural Networks und Audiosignalen in Form von Spektrogrammen die Emotionen der Menschen klassifiziert werden. Dafür wird ein Modell und die Architektur näher unter die Lupe genommen und der Zyklus bei einem SER-System erläutert.
\newline \newline\newline
\textbf{Keywords} Speech emotion recognition \textbullet Convolutional neural networks \textbullet Spektrogramm
